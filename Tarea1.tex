% Options for packages loaded elsewhere
\PassOptionsToPackage{unicode}{hyperref}
\PassOptionsToPackage{hyphens}{url}
%
\documentclass[
  12pt,
  letterpaper,
]{article}
\usepackage{amsmath,amssymb}
\usepackage{lmodern}
\usepackage{iftex}
\ifPDFTeX
  \usepackage[T1]{fontenc}
  \usepackage[utf8]{inputenc}
  \usepackage{textcomp} % provide euro and other symbols
\else % if luatex or xetex
  \usepackage{unicode-math}
  \defaultfontfeatures{Scale=MatchLowercase}
  \defaultfontfeatures[\rmfamily]{Ligatures=TeX,Scale=1}
\fi
% Use upquote if available, for straight quotes in verbatim environments
\IfFileExists{upquote.sty}{\usepackage{upquote}}{}
\IfFileExists{microtype.sty}{% use microtype if available
  \usepackage[]{microtype}
  \UseMicrotypeSet[protrusion]{basicmath} % disable protrusion for tt fonts
}{}
\makeatletter
\@ifundefined{KOMAClassName}{% if non-KOMA class
  \IfFileExists{parskip.sty}{%
    \usepackage{parskip}
  }{% else
    \setlength{\parindent}{0pt}
    \setlength{\parskip}{6pt plus 2pt minus 1pt}}
}{% if KOMA class
  \KOMAoptions{parskip=half}}
\makeatother
\usepackage{xcolor}
\usepackage[margin = 1in]{geometry}
\usepackage{color}
\usepackage{fancyvrb}
\newcommand{\VerbBar}{|}
\newcommand{\VERB}{\Verb[commandchars=\\\{\}]}
\DefineVerbatimEnvironment{Highlighting}{Verbatim}{commandchars=\\\{\}}
% Add ',fontsize=\small' for more characters per line
\usepackage{framed}
\definecolor{shadecolor}{RGB}{248,248,248}
\newenvironment{Shaded}{\begin{snugshade}}{\end{snugshade}}
\newcommand{\AlertTok}[1]{\textcolor[rgb]{0.94,0.16,0.16}{#1}}
\newcommand{\AnnotationTok}[1]{\textcolor[rgb]{0.56,0.35,0.01}{\textbf{\textit{#1}}}}
\newcommand{\AttributeTok}[1]{\textcolor[rgb]{0.77,0.63,0.00}{#1}}
\newcommand{\BaseNTok}[1]{\textcolor[rgb]{0.00,0.00,0.81}{#1}}
\newcommand{\BuiltInTok}[1]{#1}
\newcommand{\CharTok}[1]{\textcolor[rgb]{0.31,0.60,0.02}{#1}}
\newcommand{\CommentTok}[1]{\textcolor[rgb]{0.56,0.35,0.01}{\textit{#1}}}
\newcommand{\CommentVarTok}[1]{\textcolor[rgb]{0.56,0.35,0.01}{\textbf{\textit{#1}}}}
\newcommand{\ConstantTok}[1]{\textcolor[rgb]{0.00,0.00,0.00}{#1}}
\newcommand{\ControlFlowTok}[1]{\textcolor[rgb]{0.13,0.29,0.53}{\textbf{#1}}}
\newcommand{\DataTypeTok}[1]{\textcolor[rgb]{0.13,0.29,0.53}{#1}}
\newcommand{\DecValTok}[1]{\textcolor[rgb]{0.00,0.00,0.81}{#1}}
\newcommand{\DocumentationTok}[1]{\textcolor[rgb]{0.56,0.35,0.01}{\textbf{\textit{#1}}}}
\newcommand{\ErrorTok}[1]{\textcolor[rgb]{0.64,0.00,0.00}{\textbf{#1}}}
\newcommand{\ExtensionTok}[1]{#1}
\newcommand{\FloatTok}[1]{\textcolor[rgb]{0.00,0.00,0.81}{#1}}
\newcommand{\FunctionTok}[1]{\textcolor[rgb]{0.00,0.00,0.00}{#1}}
\newcommand{\ImportTok}[1]{#1}
\newcommand{\InformationTok}[1]{\textcolor[rgb]{0.56,0.35,0.01}{\textbf{\textit{#1}}}}
\newcommand{\KeywordTok}[1]{\textcolor[rgb]{0.13,0.29,0.53}{\textbf{#1}}}
\newcommand{\NormalTok}[1]{#1}
\newcommand{\OperatorTok}[1]{\textcolor[rgb]{0.81,0.36,0.00}{\textbf{#1}}}
\newcommand{\OtherTok}[1]{\textcolor[rgb]{0.56,0.35,0.01}{#1}}
\newcommand{\PreprocessorTok}[1]{\textcolor[rgb]{0.56,0.35,0.01}{\textit{#1}}}
\newcommand{\RegionMarkerTok}[1]{#1}
\newcommand{\SpecialCharTok}[1]{\textcolor[rgb]{0.00,0.00,0.00}{#1}}
\newcommand{\SpecialStringTok}[1]{\textcolor[rgb]{0.31,0.60,0.02}{#1}}
\newcommand{\StringTok}[1]{\textcolor[rgb]{0.31,0.60,0.02}{#1}}
\newcommand{\VariableTok}[1]{\textcolor[rgb]{0.00,0.00,0.00}{#1}}
\newcommand{\VerbatimStringTok}[1]{\textcolor[rgb]{0.31,0.60,0.02}{#1}}
\newcommand{\WarningTok}[1]{\textcolor[rgb]{0.56,0.35,0.01}{\textbf{\textit{#1}}}}
\usepackage{graphicx}
\makeatletter
\def\maxwidth{\ifdim\Gin@nat@width>\linewidth\linewidth\else\Gin@nat@width\fi}
\def\maxheight{\ifdim\Gin@nat@height>\textheight\textheight\else\Gin@nat@height\fi}
\makeatother
% Scale images if necessary, so that they will not overflow the page
% margins by default, and it is still possible to overwrite the defaults
% using explicit options in \includegraphics[width, height, ...]{}
\setkeys{Gin}{width=\maxwidth,height=\maxheight,keepaspectratio}
% Set default figure placement to htbp
\makeatletter
\def\fps@figure{htbp}
\makeatother
\setlength{\emergencystretch}{3em} % prevent overfull lines
\providecommand{\tightlist}{%
  \setlength{\itemsep}{0pt}\setlength{\parskip}{0pt}}
\setcounter{secnumdepth}{5}
\usepackage{longtable}
\usepackage[utf8]{inputenc}
\usepackage[spanish]{babel}\decimalpoint
\setlength{\parindent}{1.25cm}
\usepackage{amsmath}
\usepackage{xcolor}
\usepackage{cancel}
\usepackage{array}
\usepackage{float}
\usepackage{multirow}
\usepackage{booktabs}
\usepackage{longtable}
\usepackage{array}
\usepackage{multirow}
\usepackage{wrapfig}
\usepackage{float}
\usepackage{colortbl}
\usepackage{pdflscape}
\usepackage{tabu}
\usepackage{threeparttable}
\usepackage{threeparttablex}
\usepackage[normalem]{ulem}
\usepackage{makecell}
\usepackage{xcolor}
\ifLuaTeX
  \usepackage{selnolig}  % disable illegal ligatures
\fi
\IfFileExists{bookmark.sty}{\usepackage{bookmark}}{\usepackage{hyperref}}
\IfFileExists{xurl.sty}{\usepackage{xurl}}{} % add URL line breaks if available
\urlstyle{same} % disable monospaced font for URLs
\hypersetup{
  hidelinks,
  pdfcreator={LaTeX via pandoc}}

\author{}
\date{\vspace{-2.5em}}

\begin{document}

\begin{titlepage}
   \Large{
   \begin{center}
       \vspace*{1cm}

       \textbf{Tarea 1}

            
       \vspace{1.1cm}
       
       Estudiantes
       
       \vspace{0.5cm}
        
	     \textbf{John Daniel hoyos Arias} \\

       \textbf{Ivan Santiago Rojas Martinez} \\
       
       \textbf{Genaro Alfonso Aristizabal Echeverri}



              \vspace{1cm}
       
       Docente
       
       \vspace{0.5cm}

       \textbf{Juan Carlos Salazar Uribe}
       
       \vspace{0.4cm}

       \vspace{1.4cm}
       
       Asignatura
       
       \vspace{0.5cm}

       \textbf{Analitica de datos}

       \vfill

            
       \vspace{0.4cm}
     
       \includegraphics[width=0.4\textwidth]{DocumentFormat/logounal.png}
            
       Sede Medellín\\
       17 de septiembre del 2022
       
   \end{center}
   }
\end{titlepage}
\thispagestyle{empty}
\tableofcontents
\newpage
\thispagestyle{empty}
\listoffigures
\newpage

\pagestyle{myheadings}
\setcounter{page}{4}

\section{Ejercicio1}
\subsection{Clasificador de Bayes un gold estándar}

El clasificador de Bayes produce la menor tasa de error de prueba.
Demostración:

Se tiene que la tasa de error de prueba está dada por:
\[Average(I(y_0 \neq \hat{y}_0))=E[I(y_0 \neq \hat{y}_0)]\]

Se supone que se está trabajando con dos clases (Aunque también se puede
generalizar para \(n\) clases). Donde se suponen los siguientes datos de
prueba:

\[ y_{0i} = \left\lbrace \begin{array}{ll}
0 ~, ~ i=1,2,...,n;~Siendo~x_{0i}~los~datos~de~prueba \\
1
\end{array}
\right.\]

Y su respectivo clasificador de Bayes está dado por

\[P(y_{0i} = j |X_{0i}= x_{01}),~ con ~ j = 0,1\]

Donde
\[\underbrace{P(y_{0i} = 0 | x_{0i})}_{P_0} ~ > ~ \acute{o} ~ < ~ \underbrace{P(y_{0i} = 1 |x_{0i})}_{P_1}\]

Se toma la máxima probabilidad de las dos probabilidades condicionales
anteriores, es decir \(máx\{P_0, P_1\}\)

El error se minimiza cuando \(máx\{P_0, P_1\}\) lleva a aciertos, el
cual se reduce al minimizar la tasa de error promedio cuando se usa una
indicadora (Clasificador):

La siguiente igualdad es un resultado de probabilidad (Solo se cumple
para la indicadora):

\[\underbrace{E[I(\underbrace{y_{0i}\neq \widehat{y}_{0i}) }_{A})|x_{0i}]}_{Se~ desea~ minimizar}=P(y_{0i}\neq \widehat{y}_{0i}|x_{0i})=0*P(I_A=0)+1*(I_A=1)\]

El menor valor que puede obtener es:

\(E[I_{A}|x_{0i}] = 0\) que sucede cuando \(y_{0i}=\hat{y}_{0i}~~\)
\(\Longrightarrow~~\) \((y_{0i} - \hat{y}_{0i}) = 0\)
\(~~\Longrightarrow~~\) \((y_{0i} - \hat{y}_{0i})^2=0\)

Seguidamente, se intentará probar que \(y_{0i} = \hat{y_{0i}}\) produce
el menor error Cuando se usa el método de Bayes.

Sea \(f(y) = (y - \hat{y})^2\). Donde \(E[f(y)] = E[(y - \hat{y})^2]\)
es el \(MSE\), que es mínimo justamente cuando \(y_i\) es el promedio de
los errores.

Posteriormente, se utiliza una variable binaria que indiqué cuando se
comete o no un error:

\(I\{y_{0i} \neq \hat{y}_{01}\}\) el cual toma valores de
\(\left\{ \begin{aligned} 0\\ 1\\ \end{aligned} \right.\) al igual que
\(y_{0i}\) y el clasificador \(\hat{y}_{0i}\).

Entonces \((y_{0i} - \hat{y}_{0i})^2 = I\{y_{0i} \neq \hat{y}_{0i}\}\),
esta igualdad se prueba a continuación:

sea:

\[y_{0i} = I\{y_{0i} = j\}~~ y ~~ \hat{y}_{0i} = I\{\hat{y}_{0i} = j\} \longrightarrow ~~ ambas~ son ~indicadoras.\]

Utilizando la función de perdida

\[\delta(x) = (y_{0i} - \hat{y}_{0i})^2 = [I\{y_{0i} = j\} - I\{\hat{y}_{0i} = j\}]^2\]

Se prueba que:

\[ \begin{array}{ c | c }
[I\{y_{0i} = j\} - I\{\hat{y}_{0i} = j\}]^2 & I\{y_{0i} \neq\ \hat{y}_{0i}\} \\ 
\hline
(1 - 1)^2 = 0 & 0 \\ 
(1 - 0)^2 = 1 & 1  \\ 
(0 - 1)^2 = 1 & 1  \\ 
(0 - 0)^2 = 0 & 0  \\
\end{array} \]

Así que \((y_{0i} - \hat{y}_{0i})^2\) es equivalente a
\(I\{y_{0i} \neq \hat{y}_{0i} \}\)

Este proceso de clasificación esta basado en la regla de clasificación
de Bayes por lo tanto minimiza \((y_{0i} - \hat{y}_{0i})^2\) y minimiza
la tasa de error de prueba.

Esta expresión anterior se demostrará a continuación:

Se define la función de pérdida posterior como:

\[ L(y_{0i},\delta(x)) = (y_{0i}-\hat{y}_{0i})^2 = [I\{y_{0i}=j\}-I\{\hat{y}_{0i}=j\}]^2 \]
Consideremos \(y = y_{0i}\), \(x= x_{0i}\) para facilitar el proceso
algebráico a continuación:

\[ L(y,\delta(x)) = (y-\hat{y})^2 = [I\{y=j\}-I\{\hat{y}=j\}]^2 \]

Sea la función de pérdida posterior esperada:

\[
\begin{aligned}
\gamma(y,\delta(x)) & = E[L(y,\delta(x))|X=x]\\
& = E[(y-\hat{y})^2|X=x]\\
& = E[(y-E[y|x]+E[y|x]-\hat{y})^2|X=x]\\
& = E[(y-E[y|x])^2 + (E[y|x]-\hat{y})^2 + 2(y-E[y|x])(E[y|x]-\hat{y})|X=x]\\
& = E[(y-E[y|x])^2|X=x]+(E[y|x]-\hat{y})^2 + 2 
\underbrace{E[(y-E[y|x])(E[y|x]-\hat{y})|X=x]}_{\blacksquare}
\end{aligned}
\]

\[
\begin{aligned}
\blacksquare E[(y-E[y|x])(E[y|x]-\hat{y})|X=x] & = E[y \cdot E[y|x] - y\cdot\hat{y}-E^2[y|x] + \hat{y} \cdot E[y|x]|X=x]\\
& = \color{blue}{E[y|x]E[y|x]} - \color{red}{\hat{y}E[y|x]}-\color{blue}{E^2[y|x]} + \color{red}{\hat{y}E[y|x]}\\
& = 0
\end{aligned}
\]

Por lo tanto

\[\gamma(y, \delta(x))  = E[(y-E[y|x])^2|X=x]+(E[y|x]-\hat{y})^2
\geq \underbrace{E[(y-E[y|x])^2|X=x]}_{Cota~inferior:~mínimo}\]

Por lo tanto el mínimo valor que toma \(\gamma(y, \delta(x))\) es:

\[E[(y-E[y|x])^2|X=x] ~ \text{que es cuando}~ (E[y|x]-\hat{y})^2 = 0\]

Dado que \(E[(y-E[y|x])^2|X=x]\) es la cota mínima, se prueba que esta
utiliza el estimador de Bayes para estimar \(\hat{y}\) y lo esta
utilizando mediante \(E[y|x]: (E[(y- \color{purple}{E[y|x]})^2|X=x])\)
donde:

\[\widehat{y}=E[y|x]=\underbrace{P(y|x)}_{Estimador ~ de~  Bayes}\]

\section{Ejercicio2}
\subsection{Analisis Descriptivo}

\begin{tabular}{l|l|l}
\hline
  & x & Tiempo\\
\hline
 & Min.   :  0 & Min.   :  439.6\\
\hline
 & 1st Qu.: 90 & 1st Qu.: 5440.6\\
\hline
 & Median :180 & Median :11804.3\\
\hline
 & Mean   :180 & Mean   :11047.4\\
\hline
 & 3rd Qu.:270 & 3rd Qu.:16101.2\\
\hline
 & Max.   :360 & Max.   :22600.5\\
\hline
\end{tabular}

\begin{center}\includegraphics{Tarea1_files/figure-latex/unnamed-chunk-2-1} \end{center}

\begin{center}\includegraphics{Tarea1_files/figure-latex/unnamed-chunk-3-1} \end{center}

\textbf{Matriz de Autocorrelación}

\begin{center}\includegraphics{Tarea1_files/figure-latex/unnamed-chunk-4-1} \end{center}

\subsection{Modelo Parametrico}

\textbf{Modelo lineal}

\[Y = \beta_0 + \beta_1x+\epsilon \: ; \: \epsilon \sim N(0, \sigma^2)\]

\begin{verbatim}
## 
## Call:
## lm(formula = db$y ~ db$x)
## 
## Residuals:
##    Min     1Q Median     3Q    Max 
## -984.8 -227.0 -162.5  268.5  756.9 
## 
## Coefficients:
##             Estimate Std. Error t value Pr(>|t|)    
## (Intercept)  171.420    262.566   0.653    0.527    
## db$x          60.422      1.238  48.816 3.27e-14 ***
## ---
## Signif. codes:  0 '***' 0.001 '**' 0.01 '*' 0.05 '.' 0.1 ' ' 1
## 
## Residual standard error: 500.9 on 11 degrees of freedom
## Multiple R-squared:  0.9954, Adjusted R-squared:  0.995 
## F-statistic:  2383 on 1 and 11 DF,  p-value: 3.271e-14
\end{verbatim}

\begin{center}\includegraphics{Tarea1_files/figure-latex/unnamed-chunk-6-1} \end{center}

\subsection{Modelo No Parametrico}

\textbf{Test de Spearmn}

El test de Spearman es la contra parte no parametrica del test de
correlación de Pearson, ambos buscan encontrar y cuantificar el grado de
relación lineal entre dos variables. partiendo de lo anterior, como
vimos en la sección anterior, las variables x e y presentan un muy buen
ajuste lineal que deriva en una dependencia bastante marcada, el test de
Spearman nos permitirá corroborar esta dependencia y adicionalmente, nos
permitiría calcular un valor para la correlación entre las mismas. como
veremos a continuación.

Prueba unilateral derecha:

\begin{itemize}
\tightlist
\item
  \(H_0:\) X e Y son mutuamente independientes.
\item
  \(H_a:\) Existe una tendencia a formar parejas entre los valores
  grandes de X e Y.
\end{itemize}

Estadístico de prueba: El test de Pearson, presenta dos tipos de
estadísticos de prueba, uno cuando existen repeticiones entre las
observaciones, y otro en caso contrario, para el estudio que estamos
llevando a cabo se verifico dicha situación, y nos encontramos con que
efectivamente no tenemos ninguna observación repetida, por lo cual el
siguiente es el estadístico de

Prueba a utilizar:

\(\rho = 1 - \frac{6T}{n(n^2-1)}\) donde
\(T = \sum_{i=1}^{n} [R(x_i)-R(y_i)]^2\)

Criterio de rechazo: \(Rc=[\rho/\rho <- W\alpha]\) donde \(W\alpha\)
valor tabulado en la tabla A.10 con la aproximación normal.
\(VP = P(Z < \rho\sqrt{n-1})\) para un valor de \(\alpha = 0.05\) se
rechaza \(H_0\) si el Valor p \(< \alpha\)

\begin{verbatim}
## 
##  Spearman's rank correlation rho
## 
## data:  db$x and db$y
## S = 0, p-value < 2.2e-16
## alternative hypothesis: true rho is greater than 0
## sample estimates:
## rho 
##   1
\end{verbatim}

Para un valor de \(\alpha = 0.05\) existe suficiente evidencia para
rechazar \(H_0\), por lo cual se concluye que existe una fuerte
dependencia positiva entre X e y, con unvalor de \(\rho = 1\) que indica
que es completamente lineal.

\section{Ejercicio3}
\subsection{K-nearest neighbors (KNN)}

\[Pr(Y = J \mid X = x_0) \approx \frac{1}{K} \sum_{i \in N_0}I(y_i=j)\]

\begin{table}[H]

\caption{\label{tab:unnamed-chunk-9}Base de datos}
\centering
\begin{tabular}[t]{l|l|l|l}
\hline
X1 & X2 & X3 & Y\\
\hline
0 & 3 & 0 & Red\\
\hline
2 & 0 & 0 & Red\\
\hline
0 & 1 & 3 & Red\\
\hline
0 & 1 & 2 & Green\\
\hline
-1 & 0 & 1 & Green\\
\hline
1 & 1 & 1 & Red\\
\hline
\end{tabular}
\end{table}

\subsection{a) Distancia a cada observación}

Usando la distancia euclidiana entre dos punto \(u\) y \(v\) definida
como:

\[d(u, v) = \sqrt{(u_1- v_1)^2+(u_2-v_2)^2+(u_3-v_3)^2}\] Calculamos la
distancia entre cada observación y el punto \(P\) con características
\(X_1 = X_2 = X_3 = 0\)

\begin{itemize}
\item
  \(d(p, x_1) = \sqrt{(p_1- x_{1,1})^2+(p_2-x_{1,2})^2+(p_3-x_{1,3})^2} = \sqrt{(0-0)^2 + (0-3)^2 +(0-0)^2} = \sqrt{0 + 9 + 0} = \sqrt{9} = 3\)
\item
  \(d(p, x_2) = \sqrt{(p_1- x_{2,1})^2+(p_2-x_{2,2})^2+(p_3-x_{2,3})^2} = \sqrt{(0-2)^2 + (0-0)^2 + (0-0)^2} = \sqrt{4 + 0 + 0} = \sqrt{4} = 2\)
\item
  \(d(p, x_3) = \sqrt{(p_1- x_{3,1})^2+(p_2-x_{3,2})^2+(p_3-x_{3,3})^2} = \sqrt{(0-0)^2 + (0-1)^2 + (0-3)^2} = \sqrt{0 + 1 + 9} = \sqrt{10} = 3.162278\)
\item
  \(d(p, x_4) = \sqrt{(p_1- x_{4,1})^2+(p_2-x_{4,2})^2+(p_3-x_{4,3})^2} = \sqrt{(0-0)^2 + (0-1)^2 + (0-2)} = \sqrt{0 + 1 + 4} = \sqrt{5} = 2.236068\)
\item
  \(d(p, x_5) = \sqrt{(p_1- x_{5,1})^2+(p_2-x_{5,2})^2+(p_3-x_{5,3})^2} = \sqrt{(0+1)^2 + (0-0)^2 + (0-1)^2} = \sqrt{1 + 0 + 1} = \sqrt{2} = 1.414214\)
\item
  \(d(p, x_6) = \sqrt{(p_1- x_{6,1})^2+(p_2-x_{6,2})^2+(p_3-x_{6,3})^2} = \sqrt{(0-1)^2 + (0-1)^2 + (0-1)^3} = \sqrt{1 + 1 +1} = \sqrt{3} = 1.732051\)
\end{itemize}

Ahora procedemos a calcularla con R:

\begin{Shaded}
\begin{Highlighting}[]
\NormalTok{point }\OtherTok{\textless{}{-}} \FunctionTok{c}\NormalTok{(}\DecValTok{0}\NormalTok{, }\DecValTok{0}\NormalTok{, }\DecValTok{0}\NormalTok{)}

\NormalTok{dist\_eucl }\OtherTok{\textless{}{-}} \ControlFlowTok{function}\NormalTok{(x)\{}
\NormalTok{  ans }\OtherTok{\textless{}{-}} \FunctionTok{c}\NormalTok{()}
  \ControlFlowTok{for}\NormalTok{ (i }\ControlFlowTok{in} \DecValTok{1}\SpecialCharTok{:}\FunctionTok{nrow}\NormalTok{(x))\{}
\NormalTok{    xi }\OtherTok{\textless{}{-}} \FunctionTok{as.numeric}\NormalTok{(}\FunctionTok{t}\NormalTok{(}\FunctionTok{as.vector}\NormalTok{(x[i, ])))}
\NormalTok{    result }\OtherTok{\textless{}{-}} \FunctionTok{sqrt}\NormalTok{(}\FunctionTok{sum}\NormalTok{((xi}\SpecialCharTok{{-}}\NormalTok{point)}\SpecialCharTok{\^{}}\DecValTok{2}\NormalTok{))}
\NormalTok{    ans }\OtherTok{\textless{}{-}} \FunctionTok{append}\NormalTok{(ans, result)}
\NormalTok{  \}}
  \FunctionTok{return}\NormalTok{ (ans)}
\NormalTok{\}}

\NormalTok{db }\OtherTok{\textless{}{-}} \FunctionTok{mutate}\NormalTok{(db, }\AttributeTok{dist =} \FunctionTok{dist\_eucl}\NormalTok{(db[}\DecValTok{1}\SpecialCharTok{:}\DecValTok{3}\NormalTok{]))}
\end{Highlighting}
\end{Shaded}

\begin{table}[H]

\caption{\label{tab:unnamed-chunk-11}Distancia a cada observación desde el punto $X_1 = X_2 = X_3 = 0$}
\centering
\begin{tabular}[t]{r|l|r}
\hline
Observación & Grupo & Distancia Euclidiana\\
\hline
1 & Red & 3.000000\\
\hline
2 & Red & 2.000000\\
\hline
3 & Red & 3.162278\\
\hline
4 & Green & 2.236068\\
\hline
5 & Green & 1.414214\\
\hline
6 & Red & 1.732051\\
\hline
\end{tabular}
\end{table}

\subsection{b) Predicción para K = 1}

Con una selección de K = 1. Knn identifica la observación más cercana al
punto con características \(X_1 = X_2 = X_3 = 0\) y en este caso la
observación mas cercana es la \textbf{numero 5} con una distancia de
\textbf{1.414214}. Dando así Knn una estimación de \(1/1\) de pertenecer
al grupo \textbf{Green}. Por ende la estimación es pertenecer a la clase
\textbf{Green}.

Usando la librería \textbf{Class} y la funcion \textbf{knn()} se
obtiene:

\begin{Shaded}
\begin{Highlighting}[]
\FunctionTok{library}\NormalTok{(class)}
\NormalTok{point }\OtherTok{\textless{}{-}} \FunctionTok{c}\NormalTok{(}\DecValTok{0}\NormalTok{,}\DecValTok{0}\NormalTok{,}\DecValTok{0}\NormalTok{)}
\NormalTok{n }\OtherTok{\textless{}{-}} \DecValTok{6} 

\NormalTok{model }\OtherTok{\textless{}{-}} \FunctionTok{knn}\NormalTok{(}\AttributeTok{train=}\NormalTok{db[, }\SpecialCharTok{{-}}\DecValTok{4}\NormalTok{], }\AttributeTok{test=}\NormalTok{point, }\AttributeTok{cl=}\NormalTok{db[}\DecValTok{1}\SpecialCharTok{:}\DecValTok{6}\NormalTok{, }\DecValTok{4}\NormalTok{], }\AttributeTok{k =} \DecValTok{1}\NormalTok{)}
\FunctionTok{kable}\NormalTok{(model, }\AttributeTok{col.names =} \FunctionTok{c}\NormalTok{(}\StringTok{"Predicción"}\NormalTok{))}
\end{Highlighting}
\end{Shaded}

\begin{tabular}{l}
\hline
Predicción\\
\hline
Green\\
\hline
\end{tabular}

\subsection{c) Predicción para K = 3}

Con una selección de K = 3. Knn identifica las 3 observaciones más
cercanas al punto con características \(X_1 = X_2 = X_3 = 0\) y en este
caso las observaciones mas cercana son la \textbf{numero 5}, la
\textbf{numero 6} y la \textbf{numero 2} que consisten en 2
observaciones de la clase \textbf{Red} y una observación de la clase
\textbf{Green}, dando como restulado una estimación de \(2/3\) de
pertenecer a la clase \textbf{Red} y \(1/3\) de pertenecer a la clase
\textbf{Green}. Por consiguiente se estima pertenecer a la clase
\textbf{Red}.

\begin{Shaded}
\begin{Highlighting}[]
\FunctionTok{library}\NormalTok{(class)}
\NormalTok{point }\OtherTok{\textless{}{-}} \FunctionTok{c}\NormalTok{(}\DecValTok{0}\NormalTok{,}\DecValTok{0}\NormalTok{,}\DecValTok{0}\NormalTok{)}
\NormalTok{n }\OtherTok{\textless{}{-}} \DecValTok{6} 

\NormalTok{model }\OtherTok{\textless{}{-}} \FunctionTok{knn}\NormalTok{(}\AttributeTok{train=}\NormalTok{db[, }\SpecialCharTok{{-}}\DecValTok{4}\NormalTok{], }\AttributeTok{test=}\NormalTok{point, }\AttributeTok{cl=}\NormalTok{db[,}\DecValTok{4}\NormalTok{], }\AttributeTok{k =} \DecValTok{3}\NormalTok{) }
\FunctionTok{kable}\NormalTok{(model, }\AttributeTok{col.names =} \FunctionTok{c}\NormalTok{(}\StringTok{"Predicción"}\NormalTok{))}
\end{Highlighting}
\end{Shaded}

\begin{tabular}{l}
\hline
Predicción\\
\hline
Red\\
\hline
\end{tabular}

\subsection{d) Frontera de decisión de Bayes}

Si la frontera de decisión de Bayes en este problema es altamente no
lineal, ¿esperaríamos que el mejor valor de K fuera grande o pequeño?
¿Por qué?

Cuando K empieza a crecer el modelo empieza a perder flexibilidad
tomando una forma lineal. Con un k pequeño el modelo es mas flexible.
Con esto en mente el mejor valor para k es cuando k toma un valor
pequeño.

\section{Ejercicio4}

\subsection{a) Use the read.csv() function to read the data into R. Call the loaded data college. Make sure that you have the directory set to the correct location for the data}

Inicialmente cargamos los datos de la siguiente manera:

\begin{Shaded}
\begin{Highlighting}[]
\FunctionTok{library}\NormalTok{(ISLR)}
\NormalTok{College}\OtherTok{=}\NormalTok{ISLR}\SpecialCharTok{::}\NormalTok{College}
\FunctionTok{kable}\NormalTok{(}\FunctionTok{head}\NormalTok{(College))}
\end{Highlighting}
\end{Shaded}

\begin{tabular}{l|l|r|r|r|r}
\hline
  & Private & Apps & Enroll & Top10perc & Top25perc\\
\hline
Abilene Christian University & Yes & 1660 & 721 & 23 & 52\\
\hline
Adelphi University & Yes & 2186 & 512 & 16 & 29\\
\hline
Adrian College & Yes & 1428 & 336 & 22 & 50\\
\hline
Agnes Scott College & Yes & 417 & 137 & 60 & 89\\
\hline
Alaska Pacific University & Yes & 193 & 55 & 16 & 44\\
\hline
Albertson College & Yes & 587 & 158 & 38 & 62\\
\hline
\end{tabular}

\subsection{b) Look at the data using the fix() function. You should notice that the first column is just the name of each university. We don’t really want R to treat this as data. However, it may be handy to have these names for later. Try the following commands:}

Para esta sección, mostramos la estructura de los datos, la cual, a
groso modo cuenta con datos de 777 universidades

\begin{Shaded}
\begin{Highlighting}[]
\FunctionTok{fix}\NormalTok{(College)}
\FunctionTok{rownames}\NormalTok{(College)}\OtherTok{=}\NormalTok{College[,}\DecValTok{1}\NormalTok{]}
\NormalTok{College}\OtherTok{=}\NormalTok{College [,}\SpecialCharTok{{-}}\DecValTok{1}\NormalTok{]}
\FunctionTok{fix}\NormalTok{(College)}
\end{Highlighting}
\end{Shaded}

\subsection{I) Use the summary() function to produce a numerical summary of the variables in the data set.}

Luego, calculamos el resumen estadístico general para todas las
variables de la base de datos, este resumen es muy importante, dado que
nos da una mejor visión sobre la estructura y el comportamiento de
nuestra información.

\begin{verbatim}
##  Private        Apps           Accept          Enroll       Top10perc    
##  No :212   Min.   :   81   Min.   :   72   Min.   :  35   Min.   : 1.00  
##  Yes:565   1st Qu.:  776   1st Qu.:  604   1st Qu.: 242   1st Qu.:15.00  
##            Median : 1558   Median : 1110   Median : 434   Median :23.00  
##            Mean   : 3002   Mean   : 2019   Mean   : 780   Mean   :27.56  
##            3rd Qu.: 3624   3rd Qu.: 2424   3rd Qu.: 902   3rd Qu.:35.00  
##            Max.   :48094   Max.   :26330   Max.   :6392   Max.   :96.00  
##    Top25perc      F.Undergrad     P.Undergrad         Outstate    
##  Min.   :  9.0   Min.   :  139   Min.   :    1.0   Min.   : 2340  
##  1st Qu.: 41.0   1st Qu.:  992   1st Qu.:   95.0   1st Qu.: 7320  
##  Median : 54.0   Median : 1707   Median :  353.0   Median : 9990  
##  Mean   : 55.8   Mean   : 3700   Mean   :  855.3   Mean   :10441  
##  3rd Qu.: 69.0   3rd Qu.: 4005   3rd Qu.:  967.0   3rd Qu.:12925  
##  Max.   :100.0   Max.   :31643   Max.   :21836.0   Max.   :21700  
##    Room.Board       Books           Personal         PhD        
##  Min.   :1780   Min.   :  96.0   Min.   : 250   Min.   :  8.00  
##  1st Qu.:3597   1st Qu.: 470.0   1st Qu.: 850   1st Qu.: 62.00  
##  Median :4200   Median : 500.0   Median :1200   Median : 75.00  
##  Mean   :4358   Mean   : 549.4   Mean   :1341   Mean   : 72.66  
##  3rd Qu.:5050   3rd Qu.: 600.0   3rd Qu.:1700   3rd Qu.: 85.00  
##  Max.   :8124   Max.   :2340.0   Max.   :6800   Max.   :103.00  
##     Terminal       S.F.Ratio      perc.alumni        Expend     
##  Min.   : 24.0   Min.   : 2.50   Min.   : 0.00   Min.   : 3186  
##  1st Qu.: 71.0   1st Qu.:11.50   1st Qu.:13.00   1st Qu.: 6751  
##  Median : 82.0   Median :13.60   Median :21.00   Median : 8377  
##  Mean   : 79.7   Mean   :14.09   Mean   :22.74   Mean   : 9660  
##  3rd Qu.: 92.0   3rd Qu.:16.50   3rd Qu.:31.00   3rd Qu.:10830  
##  Max.   :100.0   Max.   :39.80   Max.   :64.00   Max.   :56233  
##    Grad.Rate     
##  Min.   : 10.00  
##  1st Qu.: 53.00  
##  Median : 65.00  
##  Mean   : 65.46  
##  3rd Qu.: 78.00  
##  Max.   :118.00
\end{verbatim}

Por ejemplo, en promedio el numero de estudiantes matriculados son 780
por universidad, el costo promedio de los libros es aproximadamente
549.4 dólares, la cantidad promedio de empleados para cada universidad
es de 13441 personas, otros datos interesantes como: La razón promedio
de graduación que es de 65.46, indica que de cada 100 estudiantes
aproximadamente 66 se gradúan.

\subsection{II) Use the pairs() function to produce a scatterplot matrix of the first ten columns or variables of the data. Recall that you can reference the first ten columns of a matrix A using A[,1:10].}

En esta sección realizamos un diagrama de dispersión, con el fin de
observar el grado de asociación lineal entre las primeras 10 variables
cuantitativas, el resultado fue el siguiente:

\begin{center}\includegraphics{Tarea1_files/figure-latex/unnamed-chunk-19-1} \end{center}

Inicialmente observamos una alta asociación entre las variables número
de aplicaciones recibidas(apps) y numero de aplicaciones
aceptadas(accept), esto indica que en general, que a medida que aumenta
la recepción de aplicaciones aumenta también su aceptación, lo cual
tiene mucho sentido. De la misma manera existe una alta asociación entre
aplicaciones aceptadas(accept) y número de estudiantes
matriculados(enroll). Otra relación bastante fuerte es entre los
estudiantes nuevos que hacen parte del 10\% y 25\% superior de
secundaria, indica que hacer parte de estos porcentajes puede
incrementar la probabilidad e asistir a una universidad.

\subsection{III) Use the plot() function to produce side-by-side boxplots of Outstate versus Private.}

Luego, agregamos un gráfico entre universidad publica o privada y el
numero de matriculas fuera del estado.

\begin{center}\includegraphics{Tarea1_files/figure-latex/unnamed-chunk-20-1} \end{center}

El gráfico anterior nos muestra que aparentemente no existe una
diferencia significativa sobre el comportamiento medio para el numero de
matrículas fuera del estado para universidades públicas o privadas,
aunque se podría pensar que es un poco mayor para las privadas, pero el
traslape de las cajas no es muy evidente.

\subsection{IV) Create a new qualitative variable, called Elite, by binning the Top10perc variable. We are going to divide universities into two groups based on whether or not the proportion of students coming from the top 10 percent of their high school classes exceeds 50 percent.}

En esta sección realizamos una agrupación en una nueva variable
categórica llamada Elite, donde los estudiantes con un rendimiento
superior del 10 \% en secundaria, se agrupan en si, siempre y cuando la
universidad contenga 50 o mas de ellos, en caso contrario se agrupan en
no, el resultado fue el siguiente:

\begin{tabular}{l|r}
\hline
  & Cantidad\\
\hline
Yes & 78\\
\hline
No & 699\\
\hline
\end{tabular}

\begin{center}\includegraphics{Tarea1_files/figure-latex/unnamed-chunk-22-1} \end{center}

Gráficamente parece haber mayor cantidad de matrículas fuera del estado
para los estudiantes elite, pero esta diferencia no es muy clara, pero
da indicios muy fuertes.

\subsection{V) Use the hist() function to produce some histograms with differing numbers of bins for a few of the quantitative vari ables. You may find the command par(mfrow=c(2,2)) useful: it will divide the print window into four regions so that four plots can be made simultaneously. Modifying the arguments to this function will divide the screen in other ways.}

Los siguientes histogramas muestran la distribución de algunas de las
variables, sin embargo, no es posible concluir acerca de una posible
distribución.

\begin{center}\includegraphics{Tarea1_files/figure-latex/unnamed-chunk-23-1} \end{center}

\subsection{VI) Continue exploring the data, and provide a brief summary of what you discover.}

Continúe explorando los datos.

\begin{center}\includegraphics{Tarea1_files/figure-latex/unnamed-chunk-24-1} \end{center}

\begin{center}\includegraphics{Tarea1_files/figure-latex/unnamed-chunk-24-2} \end{center}

El siguiente gráfico representa el comportamiento de las primeras 8
variables cuan-titativas, reuniendo un conjunto de gráficos muy
importantes a la hora de hacer un análisis descriptivo como lo es
observar la correlación entre las variables, de esta manera ver si
existe relación lineal entre dichas variables.

\end{document}
